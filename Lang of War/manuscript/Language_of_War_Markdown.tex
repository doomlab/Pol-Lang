\documentclass[english,man]{apa6}

\usepackage{amssymb,amsmath}
\usepackage{ifxetex,ifluatex}
\usepackage{fixltx2e} % provides \textsubscript
\ifnum 0\ifxetex 1\fi\ifluatex 1\fi=0 % if pdftex
  \usepackage[T1]{fontenc}
  \usepackage[utf8]{inputenc}
\else % if luatex or xelatex
  \ifxetex
    \usepackage{mathspec}
    \usepackage{xltxtra,xunicode}
  \else
    \usepackage{fontspec}
  \fi
  \defaultfontfeatures{Mapping=tex-text,Scale=MatchLowercase}
  \newcommand{\euro}{€}
\fi
% use upquote if available, for straight quotes in verbatim environments
\IfFileExists{upquote.sty}{\usepackage{upquote}}{}
% use microtype if available
\IfFileExists{microtype.sty}{\usepackage{microtype}}{}

% Table formatting
\usepackage{longtable, booktabs}
\usepackage{lscape}
% \usepackage[counterclockwise]{rotating}   % Landscape page setup for large tables
\usepackage{multirow}		% Table styling
\usepackage{tabularx}		% Control Column width
\usepackage[flushleft]{threeparttable}	% Allows for three part tables with a specified notes section
\usepackage{threeparttablex}            % Lets threeparttable work with longtable

% Create new environments so endfloat can handle them
% \newenvironment{ltable}
%   {\begin{landscape}\begin{center}\begin{threeparttable}}
%   {\end{threeparttable}\end{center}\end{landscape}}

\newenvironment{lltable}
  {\begin{landscape}\begin{center}\begin{ThreePartTable}}
  {\end{ThreePartTable}\end{center}\end{landscape}}

  \usepackage{ifthen} % Only add declarations when endfloat package is loaded
  \ifthenelse{\equal{\string man}{\string man}}{%
   \DeclareDelayedFloatFlavor{ThreePartTable}{table} % Make endfloat play with longtable
   % \DeclareDelayedFloatFlavor{ltable}{table} % Make endfloat play with lscape
   \DeclareDelayedFloatFlavor{lltable}{table} % Make endfloat play with lscape & longtable
  }{}%



% The following enables adjusting longtable caption width to table width
% Solution found at http://golatex.de/longtable-mit-caption-so-breit-wie-die-tabelle-t15767.html
\makeatletter
\newcommand\LastLTentrywidth{1em}
\newlength\longtablewidth
\setlength{\longtablewidth}{1in}
\newcommand\getlongtablewidth{%
 \begingroup
  \ifcsname LT@\roman{LT@tables}\endcsname
  \global\longtablewidth=0pt
  \renewcommand\LT@entry[2]{\global\advance\longtablewidth by ##2\relax\gdef\LastLTentrywidth{##2}}%
  \@nameuse{LT@\roman{LT@tables}}%
  \fi
\endgroup}


  \usepackage{graphicx}
  \makeatletter
  \def\maxwidth{\ifdim\Gin@nat@width>\linewidth\linewidth\else\Gin@nat@width\fi}
  \def\maxheight{\ifdim\Gin@nat@height>\textheight\textheight\else\Gin@nat@height\fi}
  \makeatother
  % Scale images if necessary, so that they will not overflow the page
  % margins by default, and it is still possible to overwrite the defaults
  % using explicit options in \includegraphics[width, height, ...]{}
  \setkeys{Gin}{width=\maxwidth,height=\maxheight,keepaspectratio}
\ifxetex
  \usepackage[setpagesize=false, % page size defined by xetex
              unicode=false, % unicode breaks when used with xetex
              xetex]{hyperref}
\else
  \usepackage[unicode=true]{hyperref}
\fi
\hypersetup{breaklinks=true,
            pdfauthor={},
            pdftitle={Focus on the Target: The Role of Attentional Focus in Decisions about War and Peace},
            colorlinks=true,
            citecolor=blue,
            urlcolor=blue,
            linkcolor=black,
            pdfborder={0 0 0}}
\urlstyle{same}  % don't use monospace font for urls

\setlength{\parindent}{0pt}
%\setlength{\parskip}{0pt plus 0pt minus 0pt}

\setlength{\emergencystretch}{3em}  % prevent overfull lines

\ifxetex
  \usepackage{polyglossia}
  \setmainlanguage{}
\else
  \usepackage[english]{babel}
\fi

% Manuscript styling
\captionsetup{font=singlespacing,justification=justified}
\usepackage{csquotes}
\usepackage{upgreek}

 % Line numbering
  \usepackage{lineno}
  \linenumbers


\usepackage{tikz} % Variable definition to generate author note

% fix for \tightlist problem in pandoc 1.14
\providecommand{\tightlist}{%
  \setlength{\itemsep}{0pt}\setlength{\parskip}{0pt}}

% Essential manuscript parts
  \title{Focus on the Target: The Role of Attentional Focus in Decisions about
War and Peace}

  \shorttitle{ATTENTION IN WAR DECISIONS}


  \author{Kayla N. Jordan\textsuperscript{1}, Erin M. Buchanan\textsuperscript{2}, \& William E. Padfield\textsuperscript{2}}

  % \def\affdep{{"", "", ""}}%
  % \def\affcity{{"", "", ""}}%

  \affiliation{
    \vspace{0.5cm}
          \textsuperscript{1} University of Texas - Austin\\
          \textsuperscript{2} Missouri State University  }

  \authornote{
    Kayla N. Jordan is a Ph.D.~candidate at the University of Texas at
    Austin. Erin M. Buchanan is an Associate Professor of Quantitative
    Psychology at Missouri State University. William E. Padfield is a
    Masters degree candidate at Missouri State University.
    
    Correspondence concerning this article should be addressed to Kayla N.
    Jordan, 108 E Dean Keeton St, Austin, TX 78712. E-mail:
    \href{mailto:kaylajordan@utexas.edu}{\nolinkurl{kaylajordan@utexas.edu}}
  }


  \abstract{Choosing to start a war carries with it great consequences; therefore,
it is of utmost importance to understand what predicts support for war.
We examined how word use predicted support for military action in the
U.S. Congress. Previous research has tied political language to
legislative success, party differences, and war, while others have
examined the role of the executive and public support for war. However,
the role of the legislative in war has been understudied. The present
study hypothesized that word frequencies of function and content words
would predict support for military action in the U.S. Congress. From the
Congressional Record, speeches were obtained pertaining to the decisions
for the U.S. to take military action in Kosovo, Iraq, and Libya. We
found the use of singular third person pronouns to strongly relate to
support for war among both senators and representatives.}
  \keywords{language, war, congress, pronouns, verbs \\

    
  }





\usepackage{amsthm}
\newtheorem{theorem}{Theorem}[section]
\newtheorem{lemma}{Lemma}[section]
\theoremstyle{definition}
\newtheorem{definition}{Definition}[section]
\newtheorem{corollary}{Corollary}[section]
\newtheorem{proposition}{Proposition}[section]
\theoremstyle{definition}
\newtheorem{example}{Example}[section]
\theoremstyle{definition}
\newtheorem{exercise}{Exercise}[section]
\theoremstyle{remark}
\newtheorem*{remark}{Remark}
\newtheorem*{solution}{Solution}
\begin{document}

\maketitle

\setcounter{secnumdepth}{0}



In the last few years, numerous civil disputes worldwide, which might
threaten American interests and human rights, have spurred considerable
debate over American military intervention. In fact, throughout history,
nations were periodically faced with choices about a declaration of war.
Over the past two decades, the U.S. and its allies have faced a variety
of international threats and difficulties including possible nuclear
weapons, hostile/unfriendly nations such as Iran, and human rights
abuses and genocide in Sudan and other nations. Despite declines in
legislative control of foreign policy, the U.S. Congress still plays an
important role in deciding how the military is used by retaining the
rights to formally declare war, limit the use of military force, and
control military appropriations (Phelps \& Boylan, 2002). Previous
research examined the predictors of presidential use of military force
(Clark \& Nordstrom, 2005; Keller \& Foster, 2012) and predictors of
public support for war (Cohrs \& Moschner, 2002; Friese, Fishman,
Beatson, Sauerwein, \& Rip, 2009; McCleary, Nalls, \& Williams, 2009).
However, the predictors of legislative support of military action have
been understudied, thus, presenting an interesting opportunity for
exploration (Kriner \& Shen, 2014). In this study, we sought to
determine predictors of congressional support of military action by
using language as a predictor which is a common measure in studies of
politics (Blaxill, 2013; Crew Jr. \& Lewis, 2011; Jarvis, 2004;
Slatcher, Chung, Pennebaker, \& Stone, 2007) and conflict (Kriner \&
Shen, 2014; Leudar, Marsland, \& Nekvapil, 2004; Pennebaker, 2011).
Furthermore, we explored if the most basic and objective components of
language, word frequencies, could be used as practical predictors of
support of conflict.

\subsection{Politics and Content}\label{politics-and-content}

A wide variety of predictor and outcome measures have been previously
examined to determine the role of the executive in conflict. Clark and
Nordstrom (2005) focused on political factors influencing the
probability that an executive in a democracy would engage in conflict
and found that low levels of citizen political participation, opposing
party majority in the legislature, and greater legislative control of
foreign policy was associated with lower probability that the executive
would engage in conflict. Keller and Foster (2012) used leadership trait
analysis, which is a content analytic method developed by Hermann
(2005), to classify executives' leadership style based on their language
patterns and to examine how U.S. presidents' locus of control related to
their willingness to use military force abroad to divert attention from
domestic problems. Keller and Foster found that presidents high in
internal locus of control were more likely to engage military forces
internationally and this relationship was mediated by domestic factors,
such as the gross domestic product (GDP), indicating these presidents
could use international conflict as a diversionary tactic. Leudar et al.
(2004) used membership categorization analysis, which examines how
people use words to identify with groups as well as to orient to events,
in an exploration of how George W. Bush, Tony Blair, and Osama bin Laden
used language to frame the events surrounding 9/11 and to orient to
future action. All three leaders used language to set up an us versus
them dichotomy, distinguishing allies and enemies; Bush and Blair used
political and moral language to accomplish this separation, while bin
Laden used religious language.

Turning to research on public opinion, Cohrs and Moschner (2002)
conducted a study in Germany examining predictors of students' attitudes
toward the war in Kosovo, and they found militarism, diffuse political
support, and authoritarianism predicted support for the war. They also
found some evidence of confirmation bias whereas those who were against
the war sought out information to strengthen that belief. McCleary et
al. (2009) found similar results in a study of U.S. college students
regarding support for the war in Iraq, and they found that blind
patriotism (conceptually similar to diffuse political support),
militarism, and concern for national security predicted continuing
support for the war. A different study by Friese et al. (2009) found
that political orientation predicted support for conflict in Iraq, but
this relationship was mediated by attributions of responsibility for the
war such that those who believed or were led to believe that U.S.
leaders lied about weapons of mass destruction had much less support for
the war.

Both the role of the legislative in conflict and individual word
frequencies have been under explored in their relationship to military
conflict. Kriner and Shen (2014) studied speeches pertaining to the
course of the Iraq War in the House of Representatives and found that
antiwar rhetoric by Democrats increased as the number of casualties in
the war increased, and specifically, the number of casualties from
representatives' districts. A speech was coded as antiwar if it argued
that the initial invasion was a mistake or that troops should be
withdrawn; for instance, if the congressman discussed causalities as
unacceptably high or argued that the invasion was unjustified as Saddam
Hussein posed no immediate threat, that speech was coded as antiwar.
Furthermore, number of casualties also predicted antiwar voting by
Democrats, and antiwar rhetoric by representatives was positively
correlated with antiwar attitudes held by their constituents. In
examining war discourse, Kriner and Shen (2014) only surveyed whether
the overall content of each speech was prowar or antiwar not the
specifics of the language used. Brett et al. (2007) examined individual
word use but focused on resolutions in business conflicts. They found
that greater use of negative emotion words, such as hurt or hate, and
command words, such as ought or must, decreased the likelihood of
conflict resolution while greater use of causal words, such as because
or hence, and inhibition words, such as constrain or stop, increased the
likelihood of conflict resolution. The current study combines these
ideas by using the focus of the Kriner and Shen (2014) study, war
rhetoric in Congress, and the methods of the Brett et al. (2007) study
with word frequencies as predictors.

\subsection{Language Analysis}\label{language-analysis}

Discourse is the fusion of content and style words. Within any given
sample of language, content words answer the question of what is being
said, while style words answer the question of how it is being said.
Content words include mostly nouns, verbs, and adjectives, and style
words include mostly pronouns, prepositions, articles, conjunctions,
negations, and quantifiers (Pennebaker, 2011). The Linguistic Inquiry
and Word Count (LIWC; Pennebaker, Booth, \& Frances, 2007) is a program
developed to summarize these words and others broken down into 82
language categories. Besides style words, the LIWC measures constructs
including: a) cognitive mechanisms, such as \emph{know}, \emph{because},
and \emph{none} reflecting causation, exclusivity, and certainty, b)
social and emotional words, which include words reflecting social
processes and positive and negative emotion, c) relativity, such as
\emph{go}, \emph{down}, and \emph{until} reflecting motion, space, and
time, d) and personal concerns, which include words reflecting
achievement, money, death, and religion among others. Discourse analysis
has become a popular trend to understand psychological correlates tied
to language. Tausczik and Pennebaker (2010) reviewed over 100 articles
that used language as a basis for studying other constructs;
specifically, these studies investigated how categories in the LIWC are
related to psychological phenomena, such as attention, emotionality,
dominance, and deception.

\subsection{Attentional Focus}\label{attentional-focus}

Just as a person's gaze can illuminate where their attention is so can
the words they use. Specifically, pronouns and verb tense can
demonstrate attentional focus by indicating who or what someone is
attending to in a situation and how they are processing the situation.
Therefore, greater use of first person pronouns indicated a self focus,
third person pronouns indicated a focus on others, and verb tense
indicated whether the focus was on past, present, or future events
(Tausczik \& Pennebaker, 2010). Attentional focus in the form of
pronouns has been linked to depression (Rude, Gortner, \& Pennebaker,
2004), bullying (Kowalski, 2000), and marital satisfaction (Simmons,
Gordon, \& Chambless, 2005). Little research has examined the
attentional focus in intergroup conflict situations. Abe (2012),
examining a forum discussing the Iraq War in 2002-2003, found supporters
of the war tended to have an external focus, using more third person
pronouns, and tended to use more time related words. Matsumoto, Frank,
and Hwang (2015) also found greater use of plural third person pronouns
(i.e., \emph{we}, \emph{us}) predicted aggressive acts by groups by
examining historical texts. Based on these studies as well as previous
research on intergroup conflict, we suggest those who perceive greater
threat to the ingroup may focus more negative attention on the outgroup
and focus on past events between the groups (Meeus, Duriez,
Vanbeselaere, Phalet, \& Kuppens, 2009). The purpose of the current
studies is to determine if attentional focus is different for members of
Congress who support war measures versus those who oppose them.

\subsection{Hypotheses}\label{hypotheses}

H1: Supporters of war measures will focus on other people and will
therefore use more third person pronouns (Abe, 2012; Matsumoto et al.,
2015).

H2: Supporters of wars measures will focus on past events and will
therefore use more past tense verbs (Abe, 2012).

\section{Method}\label{method}

\subsection{Language Samples}\label{language-samples}

Linguistic frequency analysis was conducted on political speeches
gleaned from Congress. The source of language samples was the
Congressional Record, a searchable database containing a record of each
session of Congress since 1995 available at
\url{https://www.congress.gov/congressional-record}, which is maintained
by the U.S. Government Publishing Office. For this study, we searched
for pertinent speeches from January 27, 1998 to September 19, 2013.
Records were included if they pertained to U.S. relations with the
following countries: Iraq, Libya, and Kosovo (see below for explanation
of country selection). Samples were split by session date and person
speaking, and therefore, each person could be represented multiple times
in the dataset. Each file in the Congressional Record includes all
speeches from the day selected, therefore, we separated each person's
speeches by day into different files for processing. For example, a
Senator may respond back and forth with an invited guest speaker, and
all the Senators spoken words would be combined into one file for that
day. Only Senators and Representatives were included in this analysis.
These speeches were then coded for party affiliation of the
Congressperson. All processed data, as well as an \emph{R} markdown
document with data analysis scripts inline with this manuscript (Aust \&
Barth, 2017) can be found at \url{https://osf.io/r8qp2/}.

\subsection{Variables}\label{variables}

\subsubsection{Language}\label{language}

Each language sample was analyzed using the Language Inquiry and Word
Count (Pennebaker et al., 2007). We examined pronouns for Hypothesis 1
and verbs for Hypothesis 2. The pronouns category included first person
singular and plural pronouns (\emph{I, me, we}), second person pronouns
(\emph{you, your}), and third person singular and plural pronouns
(\emph{he, she, they}). The verbs category included past, present, and
future tense verbs (\emph{went, does, will}). The LIWC provides
percentages of the text that fall into these categories.

\subsubsection{Military Action}\label{military-action}

For the purpose of this study, military action was defined as military
personnel being sent into another nation to coerce the actions of that
nation. In the past 15 years, the U.S. has taken military action against
Iraq, Afghanistan, Kosovo, and Libya, although Congress did not
explicitly approve action in Afghanistan or Libya. Operational
definitions for support for war were voting records (yay, nay) on bills
authorizing military action for Iraq, Kosovo, and Libya (only voted on
in the House). These bills were House Joint Resolution 114, 107th
Congress (2002); Senate Concurrent Resolution 21, 106th Congress (1999);
and House Joint Resolution 68, 112th Congress (2011). Oppose or support
information was combined with the LIWC percentages described above.

\subsection{Data Analytic Technique}\label{data-analytic-technique}

The data collected include multiple language samples by the same senator
and are structured by both party affiliation and region of interest.
This structure was best analyzed with multilevel modeling, which allowed
us to control for the correlated error terms of senator and party. We
used the \emph{nlme} package to calculate the means and standard
deviation for each variable by voting recording (Pinheiro, Bates,
Debroy, Sarkar, \& Team, 2017). The intercept was used to predict the
dependent variable (LIWC category percent), which creates a mean score
for the dependent variable. Party affiliation and Congressperson name
were controlled as random intercept factors (Gelman, 2006). The standard
error of the estimate was translated into standard deviation by
multiplying by the square root of n for the sample. This analysis was
bootstrapped using the \emph{boot} library 1000 times, and the normal
confidence interval for the mean was calculated using this function
(Canty \& Ripley, 2017). These values were separated by voting record,
Senate/House, and country of interest. The means and confidence
intervals are presented in forest plots to show the relative percentages
for each combination. The bootstrapped standard deviation values were
used to calculate \(d_s\) values using the MOTE library with the pooled
standard deviation as the denominator (Buchanan, Valentine, \& Scofield,
2017; Lakens, 2013).

\begin{table}[tbp]
\begin{center}
\begin{threeparttable}
\caption{\label{tab:Ktable}Descriptive statistics for each dependent variable by chamber, 
          region, and military support for Kosovo}
\small{
\begin{tabular}{lccccccccc}
\toprule
Chamber & Region & DV & $M_O$ & $SD_O$ & $M_S$ & $SD_S$ & $d_s$ & $d_s$ LL & $d_s$ UL\\
\midrule
House & Kosovo & I & 1.84 & 1.16 & 2.34 & 1.61 & -0.36 & -0.63 & -0.08\\
House & Kosovo & We & 3.12 & 1.56 & 2.91 & 2.06 & 0.11 & -0.16 & 0.39\\
House & Kosovo & She/He & 0.51 & 0.54 & 0.56 & 0.71 & -0.08 & -0.35 & 0.20\\
House & Kosovo & They & 0.66 & 0.56 & 0.80 & 0.98 & -0.18 & -0.45 & 0.09\\
House & Kosovo & Past & 1.91 & 1.18 & 1.78 & 1.30 & 0.12 & -0.16 & 0.39\\
House & Kosovo & Present & 7.27 & 1.98 & 6.69 & 2.57 & 0.25 & -0.02 & 0.53\\
House & Kosovo & Future & 1.34 & 0.77 & 1.64 & 1.08 & -0.32 & -0.59 & -0.04\\
Senate & Kosovo & I & 2.19 & 1.16 & 1.96 & 1.78 & 0.15 & -0.41 & 0.71\\
Senate & Kosovo & We & 3.13 & 1.89 & 1.54 & 0.57 & 1.18 & 0.56 & 1.78\\
Senate & Kosovo & She/He & 0.44 & 0.82 & 0.47 & 0.40 & -0.05 & -0.61 & 0.51\\
Senate & Kosovo & They & 0.79 & 0.62 & 0.53 & 0.36 & 0.51 & -0.06 & 1.08\\
Senate & Kosovo & Past & 2.02 & 1.16 & 2.05 & 0.72 & -0.03 & -0.59 & 0.53\\
Senate & Kosovo & Present & 8.21 & 2.53 & 5.76 & 2.05 & 1.07 & 0.46 & 1.67\\
Senate & Kosovo & Future & 1.20 & 0.41 & 1.08 & 0.67 & 0.22 & -0.34 & 0.78\\
\bottomrule
\addlinespace
\end{tabular}
}
\begin{tablenotes}[para]
\textit{Note.} Confidence intervals for $d_s$ were calculated using 
          non-central $t$ distribution. O = Oppose, S = Support, LL = Lower Limit, UL = Upper Limit.
\end{tablenotes}
\end{threeparttable}
\end{center}
\end{table}

\begin{figure}
\centering
\includegraphics{Language_of_War_Markdown_files/figure-latex/Kpic-1.pdf}
\caption{\label{fig:Kpic}House (left) and Senate (right) bootstrapped means
and 95\% confidence interval for pronouns and verb tenses for Kosovo.}
\end{figure}

\section{Study 1A - Kosovo in the
House}\label{study-1a---kosovo-in-the-house}

In early 1998, violence erupted in the Serbian region of Kosovo between
ethnic Albanians and the Serbian government. A peace agreement later in
the year lasted until the beginning of 1999 when several Albanian
civilians were killed, prompting a resurrection of hostilities. When the
Serbian government, namely President Slobodan Milosevic, failed to
concede to allowing a NATO peacekeeping force in Kosovo during February
1999 negotiations, NATO authorized air strikes against Serbian targets.
This decision subsequently prompted debate within the U.S. Congress as
to the involvement of the U.S. military in NATO's operations in Serbia
and Kosovo (Woehrel \& Kim, 2006).

In this study, we examine this debate in the U.S. House of
Representatives to determine if members of Congress who supported U.S.
military involvement focused on people or events differently than those
who opposed it.

\section{Method}\label{method-1}

Speeches made in the House of Representatives pertaining to the use of
military force in Kosovo/Serbia were gathered from the Congressional
Record available from the U.S. Government Publishing Office. In total,
210 speeches were collected. Speeches were limited to those made in the
year preceding the vote on Senate Concurrent Resolution 21 made on April
28, 1999 to allow the President to conduct air and missile strikes
against Yugoslavia (Serbia and Montenegro). This resolution failed in
the House with 213-213 with 86\% of Democrats supporting the resolution
and 84\% of Republicans opposing. These speeches were made by 156 unique
speakers where where Republicans gave 108 speeches, Democrats gave 98
speeches, one Independent, one Non-Partisan, and two
non-Representatives. Five speeches were excluded for no voting record.
The average word count was 700.51 (\emph{SD} = 814.04).

\section{Results}\label{results}

A forest plot of the results can be found in Figure \ref{fig:Kpic}, and
all descriptive statistics can be found in Table \ref{tab:Ktable}. A
small effect emerged for first-person singular pronouns and future tense
verbs. Members of Congress who supported U.S. military action tended to
use slightly more self-references and references to future actions.

\section{Study 1B - Kosovo in the
Senate}\label{study-1b---kosovo-in-the-senate}

In the second part of this study, we examined the Kosovo debate in the
U.S. Senate to determine if the differences found in the first part of
the study were also evident in the Senate.

\section{Method}\label{method-2}

Speeches were gathered in the same manner as in the first part of the
study. All speeches made in the Senate in the year before the March 23,
1999 vote on Senate Concurrent Resolution 21. This resolution passed the
Senate with 58 supporting and 41 opposing. All but 3 Democrats supported
the resolution while 70\% of Republicans opposed it. A total of 49
speeches were collected. These speeches were made by 25 unique senators
with 12 speeches by Democrats and 37 by Republicans. The average word
count for these speeches was 1413.14 (\emph{SD} = 1076.37).

\section{Results}\label{results-1}

Analyses were conducted in the same manner as the first part of the
study with bootstrapped means and CIs calculated for the seven
categories marking attention. Results can be seen as a forest plot in
Figure \ref{fig:Kpic} and Table \ref{tab:Ktable}. Sizable differences
were found in the use of first-person plural pronouns, third-person
plural pronouns, and present-tense verbs. Senators who opposed U.S.
military involvement in Kosovo tended make more group-references both to
their own group and the outgroup. Senators opposed to the legislation
also tended to make more reference to current actions.

\section{Discussion}\label{discussion}

The results of this first study are inconsistent and contrary to our
hypotheses. The results were inconsistent in that effects found for the
House and Senate are non-overlapping. For the House, supporters of war
used more first person and future tense verbs, while opposition in the
Senate used more third person and present tense verbs. It is difficult
to know exactly why this is the case; however there are several possible
explanations. First, voting in Congress is exceedingly complex and is
influenced by much more than floor debates in a given chamber. In this
case, the Senate vote on the resolution occurred before the main debate
in the House, which may have influenced what the debate focused on.
Second, the Senate and the House are composed differently. Members of
the House serve two year terms while Senators serve six year terms.
Furthermore, Senators typically have more political experience than
members of the House. These, as well as other factors, may help explain
the differential effects for the two chambers of Congress.

The results of the second part of this study were also contrary to our
hypotheses. At least in the Senate, those who supported taking military
action used fewer third person plural pronouns while there was no
difference in third person singular pronouns. Those who supported
military action also used fewer third person singular pronouns. This
finding suggests that those who opposed military action focused on both
on their ingroup and on the outgroup. Based on the findings of Abe
(2012) and Matsumoto et al. (2015), we expected those who supported
military action to show this focus. However, the results could be
explained by the situation posed by the particular resolution. In this
conflict, rather than responding to an act of aggression or a perceived
threat, the U.S. was deciding the extent to which the U.S. would be
involved in ongoing NATO, a treaty organization of which the U.S. is a
member, operations in Kosovo and Serbia. It is possible that some viewed
the outgroup as NATO rather than Serbians. In this case, with no clear,
immediate threat to the U.S., for those making ingroup-outgroup
distinctions, protecting the ingroup may have meant opposing the war
rather than supporting it. In order to determine if the situation
surrounding the Kosovo conflict may have impacted the first study, we
next turned to examine the Iraq War which was had more support and also
represented a possible clear threat to the U.S.

\section{Study 2A - Iraq in the
House}\label{study-2a---iraq-in-the-house}

In this next study, we examined the debate preceding the congressional
approval of the use of military force against Iraq. Regime change had
been a long-standing position of the U.S. toward Iraq following the Gulf
War; however serious military action was not considered until after the
World Trade Center attacks on September 11, 2001. In 2002, President
Bush declared Iraq part of an \enquote{axis of evil} in his State of the
Union address. Iraq's repeated violations of nuclear arms agreements,
ties to terrorist organizations, and pursuit of weapons of mass
destruction were argued by the Bush Administration to potentially pose a
major threat to U.S. national security. This prompted the debate within
Congress as to whether or not to approve President Bush's request for
military action (Katzman, 2002). These studies were used to determine if
the findings from the first study extend to a different conflict.
Specifically, in the first part of this study, we examined the debate in
the House of Representatives to determine if members of Congress who
supported taking military action used more self and future references.

\section{Method}\label{method-3}

Once again using the Government Publishing Office, we collected speeches
given in the House of Representatives pertaining to the use of U.S.
military force against Iraq in the three months before the vote on House
Joint Resolution 114 on October 10, 2002. This bill passed the House
with a 296-133 majority; with most Republicans supporting the measure
and 60\% of Democrats opposing. A total of 274 speeches were collected
representing 233 unique speakers. Of these speeches, 155 speeches were
made by Democrats, 119 were made by Republicans. The average word count
of the speeches was 742.34 (\emph{SD} = 1053.45). Four speeches were
excluded for no voting record.

\section{Results}\label{results-2}

\begin{table}[tbp]
\begin{center}
\begin{threeparttable}
\caption{\label{tab:Itable}Descriptive statistics for each dependent variable by chamber, 
          region, and military support for Iraq}
\small{
\begin{tabular}{lccccccccc}
\toprule
Chamber & Region & DV & $M_O$ & $SD_O$ & $M_S$ & $SD_S$ & $d_s$ & $d_s$ LL & $d_s$ UL\\
\midrule
House & Iraq & I & 1.66 & 1.33 & 1.90 & 2.15 & -0.13 & -0.37 & 0.11\\
House & Iraq & We & 3.01 & 1.61 & 2.76 & 1.37 & 0.17 & -0.07 & 0.41\\
House & Iraq & She/He & 0.56 & 0.56 & 1.16 & 0.92 & -0.77 & -1.02 & -0.52\\
House & Iraq & They & 0.46 & 0.51 & 0.49 & 1.36 & -0.03 & -0.27 & 0.21\\
House & Iraq & Past & 1.33 & 1.14 & 1.52 & 1.12 & -0.17 & -0.41 & 0.07\\
House & Iraq & Present & 6.33 & 1.96 & 6.35 & 1.62 & -0.01 & -0.25 & 0.23\\
House & Iraq & Future & 1.49 & 0.81 & 1.35 & 0.61 & 0.20 & -0.04 & 0.44\\
Senate & Iraq & I & 1.99 & 1.25 & 1.98 & 1.60 & 0.01 & -0.36 & 0.37\\
Senate & Iraq & We & 2.47 & 0.97 & 2.61 & 1.15 & -0.13 & -0.50 & 0.23\\
Senate & Iraq & She/He & 0.60 & 0.47 & 1.20 & 0.62 & -1.03 & -1.42 & -0.65\\
Senate & Iraq & They & 0.49 & 0.32 & 0.56 & 0.40 & -0.19 & -0.55 & 0.18\\
Senate & Iraq & Past & 1.39 & 0.63 & 1.84 & 1.22 & -0.42 & -0.79 & -0.05\\
Senate & Iraq & Present & 6.51 & 2.16 & 6.93 & 2.07 & -0.20 & -0.57 & 0.16\\
Senate & Iraq & Future & 1.47 & 0.59 & 1.29 & 0.53 & 0.32 & -0.05 & 0.68\\
\bottomrule
\addlinespace
\end{tabular}
}
\begin{tablenotes}[para]
\textit{Note.} Confidence intervals for $d_s$ were calculated using 
          non-central $t$ distribution. O = Oppose, S = Support, LL = Lower Limit, UL = Upper Limit.
\end{tablenotes}
\end{threeparttable}
\end{center}
\end{table}

\begin{figure}
\centering
\includegraphics{Language_of_War_Markdown_files/figure-latex/Ipic-1.pdf}
\caption{\label{fig:Ipic}House (left) and Senate (right) bootstrapped means
and 95\% confidence interval for pronouns and verb tenses for Iraq.}
\end{figure}

As in the first study, bootstrapped means and confidence intervals as
well as effect sizes (Cohen's \emph{d}) were calculated for speeches of
those supporting the measure versus those opposing the measure for the
following LIWC categories: first-person singular (\emph{I}),
first-person plural (\emph{we}), third-person singular (\emph{he},
\emph{she}), third-person plural (\emph{they}), past-tense,
present-tense, and future tense. Results can be seen as a forest plot in
Figure \ref{fig:Ipic} and in Table \ref{tab:Itable}. A non-zero effect
size difference emerged in the use of third-person singular pronouns.
Representatives who supported the military measure used other references
at a higher rate than those who opposed taking military action.

\section{Study 2B - Iraq in the
Senate}\label{study-2b---iraq-in-the-senate}

In the second part of this study, we examined the debate in the Senate.
We wished to determine if, like senators who opposed military action in
Kosovo, senators who opposed action against Iraq used more group
references as well as more reference to current events.

\section{Method}\label{method-4}

In this part of the study, speeches from the Senate were gathered for
the 6 months before the Senate vote on House Joint Resolution 114
conducted on October 11, 2002. The bill passed with a 77-23 majority.
All but one Republican supported the measure as did 58\% of Democrats.
In total, 138 speeches were collected representing 85 unique speakers.
Of these speeches, 74 were given by Democrats and 64 by Republicans. The
average word count for these speeches were 1991.23 (\emph{SD} =
1671.70).

\section{Results}\label{results-3}

Analyses were conducted in the same manner as the first part of the
study to determine differences between supporters and opponents of
military action in Iraq in terms of the use of first-person singular
(\emph{I}), first-person plural (\emph{we}), third-person singular
(\emph{he}, \emph{she}), third-person plural (\emph{they}), past-tense,
present-tense, and future tense. Figure \ref{fig:Ipic} displays these
results as a forest plot, and all values are in Table \ref{tab:Itable}.
A large difference was found in the use of third-person singular
pronouns as well as a smaller difference in the use of past tense verbs.
Senators who supported the military measure tended to use more other
references as well so as to be slightly more oriented to past events.

\section{Discussion}\label{discussion-1}

The results from this second study more closely matched our hypotheses.
For both the House and Senate, members of Congress who supported taking
military action used more singular third person pronouns (\emph{he},
\emph{his}) than those who opposed taking military action. Contrary to
our hypothesis, no differences were found for plural third person
pronouns (\emph{they}, \emph{theirs}) meaning those who supported taking
action made more references to others as specific individuals and not as
groups. Although this finding was not quite the result we expected,
these differences make sense in light of the situation. In the case of
the Iraq War, the threat was seen not as a group of people but rather a
single individual, Saddam Hussein. Hence, for supporters of military
action, their focus was still external as was expected (Abe, 2012;
Matsumoto et al., 2015); however, their focus was on an individual
rather than a group.

The second hypothesis was partially supported. In the Senate, those who
supported taking military action used more references to the past than
those opposed to military action. However, this difference was not found
in the House. As was stated previously, this difference in results could
be due to voting procedures or compositional differences in the House
and Senate. As a final test of our hypotheses, we examined the
Congressional debate surrounding U.S. involvement in Libya during its
2011 civil war. We might expect to find similar results to Study 1 as,
like the Kosovo war, there was less support for U.S. military
involvement as well as a lack of a perceived clear, immediate threat to
the U.S.

\section{Study 3 - Libya in the
House}\label{study-3---libya-in-the-house}

In this final study, we examine the debate in the House of
Representatives surrounding U.S. military involvement in Libya during
its revolution. In February 2011, a revolt against Libyan dictator,
Muammar Qaddafi, prompted the intervention of NATO when Qaddafi
violently suppressed all opposition. The involvement of NATO lead to
debate within Congress as to the exact role of the U.S. in military
operations in Libya and the extent of U.S involvement (Blanchard, 2011).
In examining this debate, we wished to determine if the language of
those who supported or opposed military action was similar to those of
either of the first two studies.

\section{Method}\label{method-5}

In this final study, the Congressional Record was searched for speeches
given in the House of Representatives pertaining to the debate of the
authorization of military action against Libya in the three months
before the vote on House Joint Resolution 68 on June 24, 2011. The bill
failed in the House 123-295. All but 14 Republicans voted against the
resolution while 60\% of Democrats supported the resolution. A total of
104 speeches were collected representing 76 unique speakers. Democrats
made 53 of these speeches while 51 speeches were made by Republicans.
The average word count for these speeches was 465.93 (\emph{SD} =
477.41). As the resolution failed in the House, it was not possible to
examine this debate in the Senate. Five speeches were excluded for no
voting record.

\section{Results}\label{results-4}

\begin{table}[tbp]
\begin{center}
\begin{threeparttable}
\caption{\label{tab:Ltable}Descriptive statistics for each dependent variable by chamber, 
          region, and military support for Libya}
\small{
\begin{tabular}{lccccccccc}
\toprule
Chamber & Region & DV & $M_O$ & $SD_O$ & $M_S$ & $SD_S$ & $d_s$ & $d_s$ LL & $d_s$ UL\\
\midrule
House & Libya & I & 2.47 & 1.66 & 2.31 & 1.13 & 0.11 & -0.31 & 0.53\\
House & Libya & We & 3.08 & 2.22 & 2.89 & 1.87 & 0.09 & -0.33 & 0.51\\
House & Libya & She/He & 0.61 & 0.83 & 0.64 & 0.85 & -0.04 & -0.46 & 0.38\\
House & Libya & They & 0.60 & 0.91 & 0.64 & 0.72 & -0.04 & -0.46 & 0.37\\
House & Libya & Past & 1.63 & 1.18 & 2.16 & 2.22 & -0.33 & -0.75 & 0.09\\
House & Libya & Present & 7.42 & 2.78 & 7.39 & 4.69 & 0.01 & -0.41 & 0.42\\
House & Libya & Future & 1.19 & 0.75 & 1.25 & 0.80 & -0.07 & -0.49 & 0.34\\
\bottomrule
\addlinespace
\end{tabular}
}
\begin{tablenotes}[para]
\textit{Note.} Confidence intervals for $d_s$ were calculated using 
          non-central $t$ distribution. O = Oppose, S = Support, LL = Lower Limit, UL = Upper Limit.
\end{tablenotes}
\end{threeparttable}
\end{center}
\end{table}

\begin{figure}
\centering
\includegraphics{Language_of_War_Markdown_files/figure-latex/Lpic-1.pdf}
\caption{\label{fig:Lpic}House (left) and Senate (right) bootstrapped means
and 95\% confidence interval for pronouns and verb tenses for Libya.}
\end{figure}

As in the first two studies, analyses consisted on comparing the
bootstrapped means, CIs, and effects sizes for those who supported the
military measure versus those who opposed it on the following linguistic
measures: first-person singular (\emph{I}), first-person plural
(\emph{we}), third-person singular (\emph{he}, \emph{she}), third-person
plural (\emph{they}), past-tense, present-tense, and future tense. These
results are displayed in Figure \ref{fig:Lpic} as a forest plot and in
Table \ref{tab:Ltable}. No differences emerged on any measure.

\section{Discussion}\label{discussion-2}

As might be expected given Study 1, no attentional differences between
those who supported and opposed taking military action in Libya in the
House of Representatives were found. This finding could indicate that in
situations where there is less Congressional support for military action
and no clear, immediate threat to the U.S., the difference between
support and opposition for military action is not a matter of attention
but other social and political forces.

\section{General Discussion}\label{general-discussion}

The most probable reason for these findings is the change in the
dynamics of war. Historically, the U.S. would declare war on another
nation (i.e., fighting the Germans in WWI). In WWII, a slight shift
occurred where the U.S. was fighting not only another nation but also an
ideology (Nazi Germany, Fascist Italy). With the beginning of the Cold
War, another movement happened where the U.S. did not directly fight
another nation (USSR) but instead fought indirectly with proxy wars
(Korean War, Vietnam War) while battling against enemy ideology
(Communism). After the Cold War and the fall of the Soviet Union, the
focus shifted to the United States' main conflict being the war on
terror (Matthews, 2014). Furthermore, Balas, Owsiak, and Diehl (2012)
argued that one possible motivation for war, since the end of the Cold
War, was the increased emphasis on the international norms of
democratization and humanitarianism. Hence, the use of singular third
person pronouns could reflect a focus on dictators violating human
rights as a cause for conflict (i.e., Hussein in Iraq, Milosevic in
Kosovo, and Qaddafi in Libya). Furthermore, the use of masculine
pronouns would seem to lend some support for this explanation.

\subsection{Limitations}\label{limitations}

The sample and methods used in the study, while useful, can also be
somewhat limited in scope. First, even though the Congressional Record
represents everything said on the floor of Congress, it does not
necessarily represent the entirety of Congress. Our sample incorporates
nearly 15 years in Congress. This time period encompassed seven election
cycles and at any given time, there are 100 senators and 435 congressmen
and women. While our data set likely included speeches from the more
influential senators and congressmen and women, we cannot predict voting
from those who did not speak. Furthermore, our findings regarding
masculine versus feminine pronouns could be confounded by the
under-representation of women in Congress. In the 113th Congress, women
comprised 20\% of the Senate and 18\% of the House (Manning \& Brudnick,
2014). For the years of voting records we used, there were 96 women in
Congress in 2011, 73 in 2002, and 67 in 1999 compared to 105 women in
the current Congress. Another limitation is tied to using word frequency
as an independent measure, although Tausczik and Pennebaker (2010) have
provided support for this research. Word frequency is a meaningful
measure of language, though it does fail to take into account context,
sarcasm, and other subtle aspects of language.

\subsection{Future Directions}\label{future-directions}

While word frequency is an interesting and relatively easy method of
linguistic analysis, other methods of content analysis could demonstrate
usefulness in understanding support for war. The current study focused
on the three most recent conflicts in Congress, but the next step might
be to discover if pronoun use has changed in discussions of war in the
last century. Finally, we studied the U.S. Congress because of a dearth
in the literature, and studies of legislative bodies of other countries
would be an excellent avenue to continue in this area. From the present
study, it is clear that language can be a useful tool to further our
understanding of the political process and its impact on war and peace.

\newpage

\section{References}\label{references}

\setlength{\parindent}{-0.5in} \setlength{\leftskip}{0.5in}

\hypertarget{refs}{}
\hypertarget{ref-Abe2012}{}
Abe, J. A. A. (2012). Cognitive--Affective styles associated With
position on war. \emph{Journal of Language and Social Psychology},
\emph{31}(2), 212--222.
doi:\href{https://doi.org/10.1177/0261927X12438532}{10.1177/0261927X12438532}

\hypertarget{ref-Aust2017}{}
Aust, F., \& Barth, M. (2017). papaja: Create APA manuscripts with R
Markdown. Retrieved from \url{https://github.com/crsh/papaja}

\hypertarget{ref-Balas2012}{}
Balas, A., Owsiak, A. P., \& Diehl, P. F. (2012). Demanding peace: The
impact of prevailing conflict on the shift from peacekeeping to
peacebuilding. \emph{Peace \& Change}, \emph{37}(2), 195--226.
doi:\href{https://doi.org/10.1111/j.1468-0130.2011.00743.x}{10.1111/j.1468-0130.2011.00743.x}

\hypertarget{ref-Blanchard2011}{}
Blanchard, C. M. (2011). \emph{Libya: Unrest and U.S. Policy} (pp.
1--43). Washington, DC: Library of Congress Washington DC Congressional
Research Service. Retrieved from
\url{http://www.dtic.mil/docs/citations/ADA543510}

\hypertarget{ref-Blaxill2013}{}
Blaxill, L. (2013). Quantifying the language of British politics,
1880-1910. \emph{Historical Research}, \emph{86}(232), 313--341.
doi:\href{https://doi.org/10.1111/1468-2281.12011}{10.1111/1468-2281.12011}

\hypertarget{ref-Brett2007}{}
Brett, J. M., Olekalns, M., Friedman, R., Goates, N., Anderson, C., \&
Lisco, G. C. (2007). Sticks and stones: Language, face, and online
dispute resolution. \emph{Academy of Management Journal}, \emph{50}(1),
85--99.
doi:\href{https://doi.org/10.5465/AMJ.2007.24161853}{10.5465/AMJ.2007.24161853}

\hypertarget{ref-Buchanan2017}{}
Buchanan, E. M., Valentine, K. D., \& Scofield, J. E. (2017). MOTE.
Retrieved from \url{https://github.com/doomlab/MOTE}

\hypertarget{ref-Canty2017}{}
Canty, A., \& Ripley, B. (2017). boot: Bootstrap R (S-Plus) Functions.
Retrieved from \url{https://cran.r-project.org/web/packages/boot/}

\hypertarget{ref-Clark2005}{}
Clark, D. H., \& Nordstrom, T. (2005). Democratic variants and
democratic variance: How domestic constraints shape interstate conflict.
\emph{The Journal of Politics}, \emph{67}(1), 250--270.
doi:\href{https://doi.org/10.1111/j.1468-2508.2005.00316.x}{10.1111/j.1468-2508.2005.00316.x}

\hypertarget{ref-Cohrs2002}{}
Cohrs, J. C., \& Moschner, B. (2002). Antiwar knowledge and generalized
political attitudes as determinants of attitude toward the Kosovo war.
\emph{Peace and Conflict: Journal of Peace Psychology}, \emph{8}(2),
139--155.
doi:\href{https://doi.org/10.1207/S15327949PAC0802_03}{10.1207/S15327949PAC0802\_03}

\hypertarget{ref-Crew2011}{}
Crew Jr., R. E., \& Lewis, C. (2011). Verbal style, gubernatorial
strategies, and legislative success. \emph{Political Psychology},
\emph{32}(4), 623--642.
doi:\href{https://doi.org/10.1111/j.1467-9221.2011.00832.x}{10.1111/j.1467-9221.2011.00832.x}

\hypertarget{ref-Friese2009}{}
Friese, M., Fishman, S., Beatson, R., Sauerwein, K., \& Rip, B. (2009).
Whose fault is it anyway? Political orientation, attributions of
responsibility, and support for the war in Iraq. \emph{Social Justice
Research}, \emph{22}(2-3), 280--297.
doi:\href{https://doi.org/10.1007/s11211-009-0095-2}{10.1007/s11211-009-0095-2}

\hypertarget{ref-Gelman2006}{}
Gelman, A. (2006). Multilevel (hierarchical) modeling: What it can and
cannot do. \emph{Technometrics}, \emph{48}(3), 432--435.
doi:\href{https://doi.org/10.1198/004017005000000661}{10.1198/004017005000000661}

\hypertarget{ref-Hermann2005}{}
Hermann, M. G. (2005). Assessing leadership style: A Trait analysis. In
Jerrold M. Post (Ed.), \emph{The psychological assessment of political
leaders: With profiles of saddam hussein and bill clinton} (First., pp.
178--212). Ann Arbor, MI: University of Michigan Press.

\hypertarget{ref-Jarvis2004}{}
Jarvis, S. E. (2004). Partisan patterns in presidential campaign
speeches, 1948--2000. \emph{Communication Quarterly}, \emph{52}(4),
403--419.
doi:\href{https://doi.org/10.1080/01463370409370209}{10.1080/01463370409370209}

\hypertarget{ref-Katzman2002}{}
Katzman, K. (2002). \emph{Terrorism: Near Eastern groups and state
sponsors, 2002} (pp. 1--48). Fort Belvoir, VA: Defense Acquisition Univ
Fort Belvoir VA David D Acker Library; Knowledge Repository. Retrieved
from \url{http://www.dtic.mil/docs/citations/ADA445109}

\hypertarget{ref-Keller2012}{}
Keller, J. W., \& Foster, D. M. (2012). Presidential leadership style
and the political use of force. \emph{Political Psychology},
\emph{33}(5), 581--598.
doi:\href{https://doi.org/10.1111/j.1467-9221.2012.00903.x}{10.1111/j.1467-9221.2012.00903.x}

\hypertarget{ref-Kowalski2000}{}
Kowalski, R. M. (2000). ``I was only kidding!'': Victims' and
perpetrators' perceptions of teasing. \emph{Personality and Social
Psychology Bulletin}, \emph{26}(2), 231--241.
doi:\href{https://doi.org/10.1177/0146167200264009}{10.1177/0146167200264009}

\hypertarget{ref-Kriner2014}{}
Kriner, D., \& Shen, F. (2014). Responding to war on capitol hill:
Battlefield casualties, congressional response, and public support for
the war in Iraq. \emph{American Journal of Political Science},
\emph{58}(1), 157--174.
doi:\href{https://doi.org/10.1111/ajps.12055}{10.1111/ajps.12055}

\hypertarget{ref-Lakens2013}{}
Lakens, D. (2013). Calculating and reporting effect sizes to facilitate
cumulative science: A practical primer for t-tests and ANOVAs.
\emph{Frontiers in Psychology}, \emph{4}.
doi:\href{https://doi.org/10.3389/fpsyg.2013.00863}{10.3389/fpsyg.2013.00863}

\hypertarget{ref-Leudar2004}{}
Leudar, I., Marsland, V., \& Nekvapil, J. (2004). On membership
categorization: `Us', `them' and `doing violence' in political
discourse. \emph{Discourse \& Society}, \emph{15}(2-3), 243--266.
doi:\href{https://doi.org/10.1177/0957926504041019}{10.1177/0957926504041019}

\hypertarget{ref-Manning2014}{}
Manning, J. E., \& Brudnick, I. A. (2014). \emph{Women in the United
States Congress, 1917-2014: Biographical and committee assignment
information, and listings by state and congress} (pp. 1917--2014).

\hypertarget{ref-Matsumoto2015}{}
Matsumoto, D., Frank, M. G., \& Hwang, H. C. (2015). The role of
intergroup emotions in political violence. \emph{Current Directions in
Psychological Science}, \emph{24}(5), 369--373.
doi:\href{https://doi.org/10.1177/0963721415595023}{10.1177/0963721415595023}

\hypertarget{ref-Matthews2014}{}
Matthews, M. (2014). \emph{Head strong: How psychology is
revolutionizing war}. New York, NY: Oxford University Press.

\hypertarget{ref-McCleary2009}{}
McCleary, D. F., Nalls, M. L., \& Williams, R. L. (2009). Types of
patriotism as primary predictors of continuing... \emph{Journal of
Military and Political Sociology}, \emph{37}(1), 77--94.

\hypertarget{ref-Meeus2009}{}
Meeus, J., Duriez, B., Vanbeselaere, N., Phalet, K., \& Kuppens, P.
(2009). Examining dispositional and situational effects on outgroup
attitudes. \emph{European Journal of Personality}, \emph{23}(4),
307--328. doi:\href{https://doi.org/10.1002/per.710}{10.1002/per.710}

\hypertarget{ref-Pennebaker2011}{}
Pennebaker, J. W. (2011). Using computer analyses to identify language
style and aggressive intent: The secret life of function words.
\emph{Dynamics of Asymmetric Conflict}, \emph{4}(2), 92--102.
doi:\href{https://doi.org/10.1080/17467586.2011.627932}{10.1080/17467586.2011.627932}

\hypertarget{ref-Pennebaker2007}{}
Pennebaker, J. W., Booth, R. J., \& Frances, M. E. (2007). Liwc2007:
Linguistic inquiry and word count. Austin, TX.

\hypertarget{ref-Phelps2002}{}
Phelps, G. A., \& Boylan, T. S. (2002). Discourses of war: The landscape
of congressional rhetoric. \emph{Armed Forces \& Society}, \emph{28}(4),
641--667.
doi:\href{https://doi.org/10.1177/0095327X0202800407}{10.1177/0095327X0202800407}

\hypertarget{ref-Pinheiro2017}{}
Pinheiro, J., Bates, D., Debroy, S., Sarkar, D., \& Team, R. C. (2017).
nlme: Linear and nonlinear mixed effects models. Retrieved from
\url{https://cran.r-project.org/package=nlme}

\hypertarget{ref-Rude2004}{}
Rude, S., Gortner, E.-M., \& Pennebaker, J. (2004). Language use of
depressed and depression-vulnerable college students. \emph{Cognition \&
Emotion}, \emph{18}(8), 1121--1133.
doi:\href{https://doi.org/10.1080/02699930441000030}{10.1080/02699930441000030}

\hypertarget{ref-Simmons2005}{}
Simmons, R. A., Gordon, P. C., \& Chambless, D. L. (2005). Pronouns in
marital interaction: What do ``you'' and ``I'' say about marital health?
\emph{Psychological Science}, \emph{16}(12), 932--936.
doi:\href{https://doi.org/10.1111/j.1467-9280.2005.01639.x}{10.1111/j.1467-9280.2005.01639.x}

\hypertarget{ref-Slatcher2007}{}
Slatcher, R. B., Chung, C. K., Pennebaker, J. W., \& Stone, L. D.
(2007). Winning words: Individual differences in linguistic style among
U.S. presidential and vice presidential candidates. \emph{Journal of
Research in Personality}, \emph{41}(1), 63--75.
doi:\href{https://doi.org/10.1016/j.jrp.2006.01.006}{10.1016/j.jrp.2006.01.006}

\hypertarget{ref-Tausczik2010}{}
Tausczik, Y. R., \& Pennebaker, J. W. (2010). The psychological meaning
of words: LIWC and computerized text analysis methods. \emph{Journal of
Language and Social Psychology}, \emph{29}(1), 24--54.
doi:\href{https://doi.org/10.1177/0261927X09351676}{10.1177/0261927X09351676}

\hypertarget{ref-Woehrel2006}{}
Woehrel, S., \& Kim, J. (2006). \emph{Kosovo and U.S. Policy} (pp.
1--30). Washington, DC: Library of Congress Washington DC Congressional
Resesarch Service. Retrieved from
\url{http://www.dtic.mil/docs/citations/ADA473482}






\end{document}
